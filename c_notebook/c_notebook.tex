%%% Basic document setup %%%
\documentclass[12pt, letterpaper]{report}
\title{Template}
\author{Anrich Tait}
\date{\today}


%%% Packages %%%
\usepackage{graphicx}	%%% Include images
\graphicspath{{images/}}%%% Where your images are stored (relative to main file)
\usepackage{xeCJK}		%%% Include japanese/chinese/korean
\usepackage{amsmath} 	%%% Math related
\usepackage{xcolor}
\usepackage{float}
\usepackage{geometry}
\usepackage{fancyhdr}
\usepackage{hyperref}
\usepackage{listings}
\setCJKmainfont{Source Han Sans JP}
\setCJKsansfont{Source Han Sans JP}
\definecolor{titlepagecolor}{cmyk}{1,.60,0,.40}
\definecolor{namecolor}{cmyk}{1,.50,0,.10}
\hypersetup{colorlinks=true,linkcolor=black,filecolor=magenta,urlcolor=cyan}

\begin{document}

\begin{titlepage}
\newgeometry{left=7.5cm} %defines the geometry for the titlepage
\pagecolor{titlepagecolor}
\noindent
\color{white}
\makebox[0pt][l]{\rule{1.3\textwidth}{1pt}}
\par
\noindent
\textbf{\textsf{コース:}} \textcolor{namecolor}{\textsf{C Programming}}
\vfill
\noindent
{\huge \textsf{C Programming}}
\vskip\baselineskip
\noindent
\textsf{作家: Anrich Tait}
\end{titlepage}
\restoregeometry % restores the geometry
\nopagecolor% Use this to restore the color pages to white

\begin{abstract}
	Quick explanation of document.
\end{abstract}
\tableofcontents

\chapter{Overview of C}
\section{Objectives:}
\begin{enumerate}
	\item Become familiar with general form of a C program and it's basic elements,
	\item Why you should write comments
	\item Use of data types and the differences between the various data types.
	\item How to declare variables
	\item How to change the values of variables
	\item Evaluate arithmetic expressions
	\item Read data values into a program and display them
	\item Understand strings
	\item Redirection to use files for input/output
	\item Understand the differences between runtime errors, syntax errors and logic errors, and how to debug each.
\end{enumerate}
\section{The basics:}
\begin{enumerate}
	\item Both \#define and \#include are handled by the pre-processor, this is
		why we cannot change a variable or value that has been \#define. The 
		compiler is not capable of going back to change it.
	\item Your variable names are also known as identifiers.
\end{enumerate}

\subsection{Steps in the compiling process:}

\begin{enumerate}
	\item preprocessing: Scans header files for relative prototypes. (So the compiler knows what printf is). Also looks for variables with \#define
	\item compiling: Turn code into assembly language before it is turned into 0's and 1's.
	\item assembling: Where the assembly language is turned into machine code.
	\item linking: Combines all the machine code into the final program that can be executed as your program.
\end{enumerate}

\subsection{Data Types:}
\begin{enumerate}
	\item int : short for integer. An int can be any whole number betweeen 
		-32767 and 32767.
	\item double : Basically a real number, which has an integral part and a 
		fractional part that is seperated by a decimal point. For example: 
		3.14159; 0.0005; 150.05. Scientific notation can be used for doubles,
		for example: the real number \[ 1.23 * 10^5 \] is equivalent to 123000.0
		where the exponent '5' means "move the decimal point 5 places to the right.
		In Scientific notation this number is written as 1.23E5. Read the letter 
		\textit{e} or \textit{E} as "times 10 to the power": 1.23e5 means 1.23 times
		10 to the power of 5. If the exponent has a minus sign the decimal point is
		moved to the left.
\end{enumerate}
Only use double when necessary, using the int data type is faster in most cases. Also
int computations are always precise whereas double numbers can have rounding 
errors.

\subsection{Input/Output operations and functions}
Data can be stored in memory in 2 diffrent ways:
\begin{enumerate}
	\item By assinging it to a variable
	\item By receiving input from a function like 'scanf'
\end{enumerate}

When data is assigned to a variable through the use of a function like 'scanf' 
it is known as an 'input operation'. This data needs to be input each time the 
program is executed.\\
That data is then stored in memory and can be output by using an \textbf{output operation}.

All input and output in C is handled by \textbf{input/output functions}. These 
functions are activated via a \textbf{function call}. \\
When you call a function you are basically telling it do something with the 
information you give it. The program will not continue until that functions 
says it is finished.\\\\

The most common functions are contained in the stdio.h header file. Examples 
of these functions are 'printf' for output and 'scanf' for input.\\\\

\textbf{A brief look at using 'printf()'}
The syntax for using printf is as follows: 
\begin{lstlisting}[language=Bash,framexleftmargin=5mm,frame=single,xleftmargin=18pt,numbers=left,numberstyle=\tiny]
printf("%[flags][width][.precision][len]specifier", var);
\end{lstlisting}
Below are some examples of each option in the above syntax:\\

[flags]: 
\begin{enumerate}
	\item - : left justify
\end{enumerate}

[width]: 
\begin{enumerate}
	\item (number) : minimum number of characters to be printed
\end{enumerate}

[precision]
\begin{enumerate}
	\item .number : number of digits to be written. The number gets padded by 0s if the resulting number is short
\end{enumerate}

[specifier]: denote type of output
\begin{enumerate}
	\item c : character
	\item d : integer
	\item f : flaoting point
	\item s : string
\end{enumerate}

\textbf{Some examples of printf()}
\begin{lstlisting}[language=C,framexleftmargin=5mm,frame=single,xleftmargin=18pt,numbers=left,numberstyle=\tiny]
/* Example for printf() */
#include <stdio.h>

int main(){
   printf ("Integers: %i %u \n", -3456, 3456);
   printf ("Characters: %c %c \n", 'z', 80);
   printf ("Decimals: %d %ld\n", 1997, 32000L);
   printf ("Some different radices: %d %x %o %#x %#o \n", 
   		100, 100, 100, 100, 100);
   printf ("floats: %4.2f %+.0e %E \n", 3.14159, 
   		3.14159, 3.14159);
   printf ("Preceding with empty spaces: %10d \n", 1997);
   printf ("Preceding with zeros: %010d \n", 1997);
   printf ("Width: %*d \n", 15, 140);
   printf ("%s \n", "Educative");
   return 0;
}
\end{lstlisting}

\begin{center}
\includegraphics[width=1\textwidth]{printf}
\end{center}































\chapter{C}
\textbf{Things to focus on when writing code:}
\begin{enumerate}
	\item Correctness
	\item Design
	\item Style
\end{enumerate}


\textbf{Unsorted notes:}
\begin{enumerate}
	\item Source code is compiled into machine code via gcc(compiler).
	\item Arguments are inputs to functions.
	\item Functions take arguments and result in output
	\item Types of outputs are: side effects (visual, audio output), return values (that can be used).
	\item Look at the mario program. \\
		Notice that a do while is used at the top. This checks whether the user has co-operated by inputting a number bigger than 0. If the user inputs 0 it will again ask for the width.
	\item Integers divided by integers truncate to only the decimal on the right of the "." it throws away all decimals
	\item How ever you can type cast with (type) variable name. (see the calculator program for example)
	\item 
\end{enumerate}

\chapter{Arrays}

The compiling process:

\begin{enumerate}
	\item preprocessing: Scans header files for relative prototypes. (So the compiler knows what printf is)
	\item compiling: Turn code into assembly language before it is turned into 0's and 1's.
	\item assembling: Where the assembly language is turned into machine code.
	\item linking: Combines all the machine code into the final program that can be executed as your program.
\end{enumerate}

\textbf{Debugging}

buggy.c
\begin{lstlisting}[language=C,numbers=left,framexleftmargin=5mm,frame=single,xleftmargin=18pt]
#include <stdio.h>

int main(void)
{
	for (int i = 0; i <= 3; i++)
	{
		printf("i is %i\n", i);
		printf("#\n");
	}
}
\end{lstlisting}
The code above will print a column of 3 \#.\\
The commented line is an example of printf debugging. (A method of debugging where the programmer uses printf to check the values of variables while the code runs).\\
In this example our issue was that we expected a column of 3 \# symbols but instead got 4. After using the printf debugging method it was easy to see that the "i" variable 
was incremented one to may times due to this little code section. \\
for (int i = 0; \textbf{i <= 3}; i++)


\chapter{Functions}
\section{Outline:}
	The following questions/topics will be addressed:
	\begin{enumerate}
		\item What are functions?
		\item How are functions utilised?
		\item Examples of using functions.
	\end{enumerate}

\section{Definition}
Functions break large computing tasks into smaller ones. This helps make code 
easier to change in the future and can also help  with readability. Appropriate 
functions also hide details of operations from parts of the program that don't 
need them. C programs generally consist of many small functions rather than fewer 
large ones. The most basic example of a function is main(), every program has one 
and it is executed before any other functions in the program. Functions can be 
used for any task but generally it is best for each function to accomplish a 
specific task. A function declaration provides the actual body of the function. \\

The standard library provides numerous builtin functions that your program can call
for example:
\begin{enumerate}
	\item strcat() to concatenate strings
	\item memcpy() to copy the memory location of one parameter to another.
\end{enumerate}

\clearpage
\section{Defining a function}

\subsection*{Syntax:}
\begin{lstlisting}
	return_type function_name(parameter list)
	{
		body of function
	}
\end{lstlisting}

Here is a break down of each part of this syntax:

\begin{enumerate}
	\item return\_type: The type of value the function will return.
	\item function\_name: Name of the function.
	\item (parameter list): list of parameters passed to the function.
	\item body of function: code to be executed
\end{enumerate}

Example:
\begin{lstlisting}
	int add(int a, int b)
	{
		int result = a + b;
		return (result);
	}
\end{lstlisting}
 
Break down:
\begin{enumerate}
	\item int: Return type of function (integer).
	\item add: Name of function.
	\item (int a, int b)\: parameters.
	\item int result \= a + b: declare result variable and code that will be executed.
	\item return (result)\: returns the value of the result variable.
\end{enumerate}

\textbf{Note:} if a function does not return anything it can be specified as void.

Functions can be called from other parts of a program by using the function name 
and passing in the required parameters.

\begin{lstlisting}
	int x = 5;
	int y = 7;

	int z = add(x, y);
\end{lstlisting}
The function 'add' is called with x and y as the parameters. The x parameter 
will replace 'a' and 'y' will replace 'b'. The result is stored in 'z'.


\section{Function types}

\begin{enumerate}
	\item Functions that return a value:
		Perform a specific task and return a value of a specific data type using 
		the return statement. The previous "add" function is an example of this.
	\begin{lstlisting}
		char get_first_char(char *str)
		{
			return[0];
		}
	\end{lstlisting}
	\item Functions that return nothing:
	Perform a task but does not return any value.
	\begin{lstlisting}
		void print_hello()
		{
			printf("Hello, world!")
		}
	\end{lstlisting}
	\item Functions that take no parameters:
		Does not take any paramters but perfomrs a specific task.
	\begin{lstlisting}
		void clear_screen()
		{
			system("clear");
		}
	\end{lstlisting}
	\item Functions that take parameters but return nothing.
	\begin{lstlisting}
		void greet(char *name)
		{
			printf("Hello, %s", name);
		}
	\end{lstlisting}
\end{enumerate}


\section{Calling a function}
When a function is called, control of the program is transferred to the function
until it's return statement or closing brace is reached. Then the main() 
function is used again.\\

To call a function you need to pass the required parameters.\\

Passing parameters to a function basically means providing values or variables
to the function.

Going back to the 'add' function as an example. \\
x and y are the parameters that are passed to the function. When the function is
called with arguments the values of the arguments are assigned to the corresponding
parameters. So if we call the 'add' function with arguments 3 and 4 the values
3 and 4 are assigned to x and y respectively.\\

\clearpage
\section{Function arguments}
For a function to use arguments it must declare variables that accept the values
of the arguments. These variables are called the formal parameters of a function.

Formal parameters behave like other local variables and are created upon entry
to the function and are destroyed upon exit.

There are two ways arguments can be passed to a function:

\subsection{Call/Pass by value:}
The function receives a copy of the argument passed, not the original argument.
So any changes made to the argument inside the function do not change the original
value of the argument.
		

\subsection{Call/Pass by reference:}
Calling a function by reference involves passing arguments to a function using
the parameters address. This way the actual value is changed, not just a copy
of it.\\

To pass a variable by reference in C, you must declare the function parameter 
as a pointer type using the * operator.\\


The following example function takes a pointer to an integer as a parameter.

\textbf{Note:} by default C passes arguments by value.
\clearpage


\section{Function Prototypes}

\subsection{What is a function prototype}
A function prototype is a declaration of a function that describes the function's
interface to the rest of the program. It specifies the function's name, return
type and parameter types.\\

These are usually specified at the beginning of a file before any other functions.

\subsection*{Syntax:}
\begin{lstlisting}
	return_type function_name(parameter_list);
\end{lstlisting}

For example:
\begin{lstlisting}
	int max(int numOne, int numTwo);
\end{lstlisting}
This prototype declares a funtion named max that takes two integer parameters.

\subsection{Why are function prototypes used?}

\begin{enumerate}
	\item Type checking: By declaring the function prototype before calling the 
		function the compiler can check that the arguments passed to the function 
		match the expected types.
	\item Avoid implicit declaration: If a function is not given an implicit type 
		the compiler assumes it is an integer, this can lead to misleading errors 
		relating to different return types or number of arguments.
	\item Optimization: Allows the compiler to generate more efficient code by
		providing information about the function's interface. This can lead to
		smaller, faster code.
	\item Documentation: Can server as documentation for the function, making
		it easier for other programmers to understand how to use a function.
\end{enumerate}

\chapter{Pointers}

	\subsection*{Syntax:}
	<type> *<name>

\vspace{0.25cm}

\begin{enumerate}
	\item type = is the data type that the pointer will point to.
	\item name = is the name of the pointer variable.
\end{enumerate}


\subsection*{Example:}
To declare a pointer variable that will point to an integer value the 
following syntax is used,
\begin{lstlisting}
	int *ptr;
\end{lstlisting}
This declares a pointer variable named "ptr" that can point to an integer 
value.\\

To assign  a value to a pointer variable the address-of operator (\&) is used 
to get the memory address of the variable.\\

\subsection*{Example:}
Assign the address of an intger variable named "x" to the pointer variable 
"ptr",
\begin{lstlisting}
	int x = 10;
	int *ptr = &x;
\end{lstlisting}
This assigns the address of "x" to "ptr", so now "ptr" points to x.
To access the value that is stored at the memory location pointed to by a
pointer variable the dereference operator is used (*).\\

\section{Concept}
Pointers are a fundamental concept in the C programming language, 
and understanding how they work is essential for writing efficient and 
effective C code. A pointer is a variable that stores the memory address of 
another variable. By using pointers, C programs can directly access and 
manipulate the values of other variables in memory, providing a powerful tool 
for building complex data structures and optimizing program performance.\\

At its core, a pointer is simply a memory address. When you declare a pointer 
variable in C, you are creating a variable that can store the address of 
another variable in memory. To declare a pointer variable, you use the * 
symbol before the variable name, like this: int *ptr;. This declares a pointer
variable named ptr that can store the memory address of an integer variable.\\

Once you have declared a pointer variable, you can use the \& operator to get 
the memory address of another variable, and assign that address to the pointer
variable using the assignment operator =. For example, if you have an integer 
variable named x, you can get its memory address with \&x, and assign it to the 
pointer variable ptr like this: ptr = \&x;. Now, ptr points to the memory 
location of x, and you can use the pointer to access and manipulate the value 
of x directly.\\

One of the most common uses of pointers in C is to pass variables by 
reference to functions. In C, when you pass a variable to a function as an 
argument, the function gets a copy of the variable's value. However, by 
passing the memory address of the variable instead, you can allow the function 
to directly access and modify the variable's value. This can be useful for 
functions that need to modify variables outside of their own scope.\\

Another important use of pointers in C is for dynamic memory allocation. In C,
you can use the malloc() function to allocate a block of memory of a specified 
size at runtime. The malloc() function returns a pointer to the beginning of 
the allocated memory block, which you can then use to access and manipulate 
the memory. Once you are done with the memory, you should free it using the 
free() function to avoid memory leaks.\\

Pointers can also be used to create complex data structures in C, such as 
linked lists, trees, and graphs. By using pointers to connect nodes together 
in these structures, you can create efficient and flexible data structures 
that can be easily traversed and manipulated.\\

However, while pointers are a powerful tool in C programming, they can also be
dangerous if used improperly. Pointer errors, such as dereferencing a null 
pointer or accessing memory that has already been freed, can cause program 
crashes or even security vulnerabilities. To avoid these errors, it is 
important to carefully manage and validate pointers in your code, and to use 
best practices for pointer manipulation, such as initializing pointers to null
and checking for null before dereferencing them.\\

In conclusion, pointers are a fundamental concept in C programming that allow 
you to directly access and manipulate the values of variables in memory. 
Pointers are used extensively in C for passing variables by reference, 
dynamic memory allocation, and creating complex data structures. While 
pointers can be a powerful tool, they can also be dangerous if used improperly,
and it is important to carefully manage and validate pointers in your code to 
avoid errors and vulnerabilities.\\


\subsection*{Summary:}
The syntax for a pointer consists of the data type followed by an asterisk
and the name of the pointer variable. The address-of operator (\&) is used 
to assign a value to the pointer, and the dereference operator (*) is 
used to access the value stored at the memory location pointed to by the 
pointer.

\newpage
\section{Examples}

\subsection{How to use pointers:}
In this example, swap() is a function that takes two integer pointers as 
arguments and swaps their values. By passing the memory addresses of a and b 
to swap() using the \& operator, we allow the function to directly modify the 
values of a and b.

\clearpage
In this example, we use the malloc() function to dynamically allocate an array 
of 10 integers, and store its memory address in the pointer variable arr. We 
then use a loop to assign the values 0 to 9 to the array elements. Finally, we
use the free() function to release the allocated memory.



\end{document}
