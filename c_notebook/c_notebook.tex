%%% Basic document setup %%%
\documentclass[12pt, letterpaper]{report}
\title{Template}
\author{Anrich Tait}
\date{\today}


%%% Packages %%%
\usepackage{graphicx}	%%% Include images
\graphicspath{{images/}}%%% Where your images are stored (relative to main file)
\usepackage{xeCJK}		%%% Include japanese/chinese/korean
\usepackage{amsmath} 	%%% Math related
\usepackage{xcolor}
\usepackage{float}
\usepackage{geometry}
\usepackage{fancyhdr}
\usepackage{hyperref}
\usepackage{listings}
\setCJKmainfont{Source Han Sans JP}
\setCJKsansfont{Source Han Sans JP}
\definecolor{titlepagecolor}{cmyk}{1,.60,0,.40}
\definecolor{namecolor}{cmyk}{1,.50,0,.10}
\hypersetup{colorlinks=true,linkcolor=black,filecolor=magenta,urlcolor=cyan}

\begin{document}

\begin{titlepage}
\newgeometry{left=7.5cm} %defines the geometry for the titlepage
\pagecolor{titlepagecolor}
\noindent
\color{white}
\makebox[0pt][l]{\rule{1.3\textwidth}{1pt}}
\par
\noindent
\textbf{\textsf{コース:}} \textcolor{namecolor}{\textsf{C Programming}}
\vfill
\noindent
{\huge \textsf{C Programming}}
\vskip\baselineskip
\noindent
\textsf{作家: Anrich Tait}
\end{titlepage}
\restoregeometry % restores the geometry
\nopagecolor% Use this to restore the color pages to white

\begin{abstract}
	Quick explanation of document.
\end{abstract}
\tableofcontents

\chapter{Overview of C}
\section{Objectives:}
\begin{enumerate}
	\item Become familiar with general form of a C program and it's basic elements,
	\item Why you should write comments
	\item Use of data types and the differences between the various data types.
	\item How to declare variables
	\item How to change the values of variables
	\item Evaluate arithmetic expressions
	\item Read data values into a program and display them
	\item Understand strings
	\item Redirection to use files for input/output
	\item Understand the differences between runtime errors, syntax errors and logic errors, and how to debug each.
\end{enumerate}
\textbf{Unsorted Notes:}
\begin{enumerate}
	\item Both \#define and \#include are handled by the pre-processor, this is
		why we cannot change a variable or value that has been \#define. The 
		compiler is not capable of going back to change it.
	\item Your variable names are also known as identifiers.
\end{enumerate}

\textbf{Steps in the compiling process:}

\begin{enumerate}
	\item preprocessing: Scans header files for relative prototypes. (So the compiler knows what printf is). Also looks for variables with \#define
	\item compiling: Turn code into assembly language before it is turned into 0's and 1's.
	\item assembling: Where the assembly language is turned into machine code.
	\item linking: Combines all the machine code into the final program that can be executed as your program.
\end{enumerate}

\textbf{Data Types:}
\begin{enumerate}
	\item int : short for integer. An int can be any whole number betweeen 
		-32767 and 32767.
	\item double : Basically a real number, which has an integral part and a 
		fractional part that is seperated by a decimal point. For example: 
		3.14159; 0.0005; 150.05. Scientific notation can be used for doubles,
		for example: the real number \[ 1.23 * 10^5 \] is equivalent to 123000.0
		where the exponent '5' means "move the decimal point 5 places to the right.
		In Scientific notation this number is written as 1.23E5. Read the letter 
		\textit{e} or \textit{E} as "times 10 to the power": 1.23e5 means 1.23 times
		10 to the power of 5. If the exponent has a minus sign the decimal point is
		moved to the left.
\end{enumerate}
Only use double when necessary, using the int data type is faster in most cases. Also
int computations are always precise whereas double numbers can have rounding 
errors.































\end{document}
