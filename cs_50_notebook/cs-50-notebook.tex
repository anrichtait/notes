%%% Basic document setup %%%
\documentclass[12pt, letterpaper]{report}
\title{Template}
\author{Anrich Tait}
\date{\today}


%%% Packages %%%
\usepackage{graphicx}	%%% Include images
\graphicspath{{images/}}%%% Where your images are stored (relative to main file)
\usepackage{xeCJK}		%%% Include japanese/chinese/korean
\usepackage{amsmath} 	%%% Math related
\usepackage{xcolor}
\usepackage{float}
\usepackage{geometry}
\usepackage{fancyhdr}
\usepackage{hyperref}
\usepackage{listings}
\setCJKmainfont{Source Han Sans JP}
\setCJKsansfont{Source Han Sans JP}
\definecolor{titlepagecolor}{cmyk}{1,.60,0,.40}
\definecolor{namecolor}{cmyk}{1,.50,0,.10}
\hypersetup{colorlinks=true,linkcolor=black,filecolor=magenta,urlcolor=cyan}

\begin{document}

\begin{titlepage}
\newgeometry{left=7.5cm} %defines the geometry for the titlepage
\pagecolor{titlepagecolor}
\noindent
\color{white}
\makebox[0pt][l]{\rule{1.3\textwidth}{1pt}}
\par
\noindent
\textbf{\textsf{コース:}} \textcolor{namecolor}{\textsf{CS 50}}
\vfill
\noindent
{\huge \textsf{Harvard CS50 Notebook}}
\vskip\baselineskip
\noindent
\textsf{作家: Anrich Tait}
\end{titlepage}
\restoregeometry % restores the geometry
\nopagecolor% Use this to restore the color pages to white

\tableofcontents
\begin{abstract}
My notes during the CS50 Harvard online course.
\end{abstract}

\chapter{C}
\textbf{Things to focus on when writing code:}
\begin{enumerate}
	\item Correctness
	\item Design
	\item Style
\end{enumerate}


\textbf{Unsorted notes:}
\begin{enumerate}
	\item Source code is compiled into machine code via gcc(compiler).
	\item Arguments are inputs to functions.
	\item Functions take arguments and result in output
	\item Types of outputs are: side effects (visual, audio output), return values (that can be used).
	\item Look at the mario program. \\
		Notice that a do while is used at the top. This checks whether the user has co-operated by inputting a number bigger than 0. If the user inputs 0 it will again ask for the width.
	\item Integers divided by integers truncate to only the decimal on the right of the "." it throws away all decimals
	\item How ever you can type cast with (type) variable name. (see the calculator program for example)
	\item 
\end{enumerate}

\chapter{Arrays}

The compiling process:

\begin{enumerate}
	\item preprocessing: Scans header files for relative prototypes. (So the compiler knows what printf is)
	\item compiling: Turn code into assembly language before it is turned into 0's and 1's.
	\item assembling: Where the assembly language is turned into machine code.
	\item linking: Combines all the machine code into the final program that can be executed as your program.
\end{enumerate}

\textbf{Debugging}

buggy.c
\begin{lstlisting}[language=C,numbers=left,framexleftmargin=5mm,frame=single,xleftmargin=18pt]
#include <stdio.h>

int main(void)
{
	for (int i = 0; i <= 3; i++)
	{
		printf("i is %i\n", i);
		printf("#\n");
	}
}
\end{lstlisting}
The code above will print a column of 3 \#.\\
The commented line is an example of printf debugging. (A method of debugging where the programmer uses printf to check the values of variables while the code runs).\\
In this example our issue was that we expected a column of 3 \# symbols but instead got 4. After using the printf debugging method it was easy to see that the "i" variable 
was incremented one to may times due to this little code section. \\
for (int i = 0; \textbf{i <= 3}; i++)






















\end{document}
