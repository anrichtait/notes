%%% Basic document setup %%%
\documentclass[12pt, letterpaper]{report}
\title{Template}
\author{Anrich Tait}
\date{\today}


%%% Packages %%%
\usepackage{graphicx}	%%% Include images
\graphicspath{{images/}}%%% Where your images are stored (relative to main file)
\usepackage{xeCJK}		%%% Include japanese/chinese/korean
\usepackage{amsmath} 	%%% Math related
\usepackage{xcolor}
\usepackage{float}
\usepackage{geometry}
\usepackage{fancyhdr}
\usepackage{hyperref}
\usepackage{listings}
\setCJKmainfont{Source Han Sans JP}
\setCJKsansfont{Source Han Sans JP}
\definecolor{titlepagecolor}{cmyk}{1,.60,0,.40}
\definecolor{namecolor}{cmyk}{1,.50,0,.10}
\hypersetup{colorlinks=true,linkcolor=black,filecolor=magenta,urlcolor=cyan}

\begin{document}

\begin{titlepage}
\newgeometry{left=7.5cm} %defines the geometry for the titlepage
\pagecolor{titlepagecolor}
\noindent
\color{white}
\makebox[0pt][l]{\rule{1.3\textwidth}{1pt}}
\par
\noindent
\textbf{\textsf{コース:}} \textcolor{namecolor}{\textsf{C Programming}}
\vfill
\noindent
{\huge \textsf{The C Programming Language - BK and DR Notebook}}
\vskip\baselineskip
\noindent
\textsf{作家: Anrich Tait}
\end{titlepage}
\restoregeometry % restores the geometry
\nopagecolor% Use this to restore the color pages to white

\begin{abstract}
	My notes while working through the second edition of Brian Kernighan and Dennis
	Ritchies's C Programming Language textbook.
\end{abstract}
\tableofcontents

\chapter{A tutorial introduction}
* See excercises in the external folder marked: programs

\section{Getting started}
\begin{enumerate}
	\item Nothing really new in this section.
\end{enumerate}

\section{Variables and Arithmetic Expressions}
\textbf{Notes on the fahrenheit celsius program:}
\begin{enumerate}
	\item Formula to use: Celsius = (5/9)(fahrenheit-32)
	\item This formula will be used to calculate the celsius and fahrenheit equivalents.
	\item Each line of the table is computed using a while that repeats once per output line.
	\item The while loop will test if the condition is true (fahr is less than or equal to upper.
	\item The body of the loop will then be executed.
	\item The reason for multiplying by 5 and then dividing by 9 instead of just multiplying by 5/9 is that integer division is truncated.
		(any fractional part is discarded)
\end{enumerate}

The code should look something like this

\lstinputlisting[
caption=Fahr celcius table implementation,
language=C, % Specificies language in the block
frame=single, % Frame around the code listing
showstringspaces=false, % Don't put marks in string spaces
numbers=left, % Line numbers on left
numberstyle=\tiny, % Line numbers styling
]{programs/fahr-celc.c}

The above temperature converter makes use of a while loop and 3 different variables.
This code works and outputs the expected result but can also be written many other ways.
One way to write it would be with a for loop.

\lstinputlisting[
caption=Fahr celsius table with a for loop,
language=C, % Specificies language in the block
frame=single, % Frame around the code listing
showstringspaces=false, % Don't put marks in string spaces
numbers=left, % Line numbers on left
numberstyle=\tiny, % Line numbers styling
]{programs/fahr-celc.c}

\section{Excercise 1-5}
\textbf{Task:} Modify the temperature conversion program to print the table in
reverse order, from 300 degrees to 0.

\lstinputlisting[
caption=Excericise 1-5,
language=C, % Specificies language in the block
frame=single, % Frame around the code listing
showstringspaces=false, % Don't put marks in string spaces
numbers=left, % Line numbers on left
numberstyle=\tiny, % Line numbers styling
]{programs/fahr-celc.c}

























\end{document}
