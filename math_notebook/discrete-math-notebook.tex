%%% Basic document setup %%%
\documentclass[12pt, letterpaper]{report}
\title{Template}
\author{Anrich Tait}
\date{\today}


%%% Packages %%%
\usepackage{graphicx}	%%% Include images
\graphicspath{{images/}}%%% Where your images are stored (relative to main file)
\usepackage{xeCJK}		%%% Include japanese/chinese/korean
\usepackage{amsmath} 	%%% Math related
\usepackage{amsfonts}
\usepackage{amssymb}
\usepackage{xcolor}
\usepackage{float}
\usepackage{geometry}
\usepackage{fancyhdr}
\usepackage{hyperref}
\usepackage{listings}
\setCJKmainfont{Source Han Sans JP}
\setCJKsansfont{Source Han Sans JP}
\definecolor{titlepagecolor}{cmyk}{1,.60,0,.40}
\definecolor{namecolor}{cmyk}{1,.50,0,.10}
\hypersetup{colorlinks=true,linkcolor=black,filecolor=magenta,urlcolor=cyan}

\begin{document}

\begin{titlepage}
\newgeometry{left=7.5cm} %defines the geometry for the titlepage
\pagecolor{titlepagecolor}
\noindent
\color{white}
\makebox[0pt][l]{\rule{1.3\textwidth}{1pt}}
\par
\noindent
\textbf{\textsf{コース:}} \textcolor{namecolor}{\textsf{Math self-study}}
\vfill
\noindent
{\huge \textsf{Discrete Mathematics Notebook}}
\vskip\baselineskip
\noindent
\textsf{作家: Anrich Tait}
\end{titlepage}
\restoregeometry % restores the geometry
\nopagecolor% Use this to restore the color pages to white

\tableofcontents
\begin{abstract}
My Discrete Mathematics self-study notes and excercises.
\end{abstract}

\chapter{Sets, Sequences and functions}

\section{Some Special Sets}
Set theory is used as the underlying basis for Mathematics.
Set theory defines a "set" as collection of objects. Sets are denoted using 
capital letters such as A, B, S or X.\\\\

An object that belongs to a set is a called a member/element of that set. For 
example if 'a' is an object and A is a set we can denote it by writing:\\

This is a test line $\frac{x}{y}$









\chapter{Math in latex cheat reference}

$	A \cup B		$\\
$	A \cap B		$\\
$	A \backslash B	$\\
$	A \triangle B	$\\
$	x \in S			$\\
$	x \not\in S		$\\
$	S \subseteq T	$\\
$	S \not\subseteq T$\\
$	|s|				$\\
$	\wp(S)			$\\
$	\emptyset		$\\
$	\mathbb{N}		$\\
$	\mathbb{Z}		$\\
$	\mathbb{R}		$\\
$	\aleph_0		$\\
$	\frac{1}{2}		$\\
$	\sqrt{x}		$\\
$	x^y				$\\
$	\sum			$\\
$	\Sigma			$\\
$	\geq			$\\
$	\infty			$\\
$	\leq			$\\
$	\pi				$\\
$	\emptyset		$\\
$	\oplus			$\\
$	\rightarrow		$\\
$	\int			$\\
$	\approx			$\\
$	\neq			$\\






\begin{enumerate}
	\item This is item one
	\item This is item two
	\item aa
	\item aa
	\item zz
\end{enumerate}







\end{document}
